\documentclass{report}

\usepackage[T2A]{fontenc}
\usepackage[utf8]{luainputenc}
\usepackage[english, russian]{babel}
\usepackage[pdftex]{hyperref}
\usepackage[14pt]{extsizes}
\usepackage{listings}
\usepackage{color}
\usepackage{geometry}
\usepackage{enumitem}
\usepackage{multirow}
\usepackage{graphicx}
\usepackage{indentfirst}
\usepackage{amsmath}

\geometry{a4paper,top=2cm,bottom=2cm,left=2.5cm,right=1.5cm}
\setlength{\parskip}{0.5cm}
\setlist{nolistsep, itemsep=0.3cm,parsep=0pt}

\usepackage{xcolor}
\definecolor{weborange}{RGB}{255,165,0}
\definecolor{codepurple}{RGB}{220,20,60}
\definecolor{codecyan}{RGB}{135,0,198}
\definecolor{codeblue}{RGB}{69,97,189}
\definecolor{backcolour}{RGB}{230,230,220}

\usepackage{listings}
\lstset{language=C++,
	basicstyle=\ttfamily\footnotesize,
	backgroundcolor=\color{backcolour},
    	emph={int, double, time_t, char, new},
    	emphstyle={\color{blue}},
    	commentstyle=\color{green!60!black},
 	keywordstyle={\color{codepurple}},
 	keywords=[2]{std},
    	keywordstyle=[2]{\color{weborange}},
    	otherkeywords = {:},
	morekeywords = [2]{:},
   	stringstyle=\color{codepurple},
	morecomment=[l][\color{codecyan}]{\#}, 
	tabsize=2,
	breaklines=true,                               
  	breakatwhitespace=true,
  	title=\lstname,  
}

\makeatletter
\renewcommand\@biblabel[1]{#1.\hfil}
\makeatother

\begin{document}

\begin{titlepage}

\begin{center}
Министерство науки и высшего образования Российской Федерации
\end{center}

\begin{center}
Федеральное государственное автономное образовательное учреждение высшего образования \\
Национальный исследовательский Нижегородский государственный университет им. Н.И. Лобачевского
\end{center}

\begin{center}
Институт информационных технологий, математики и механики
\end{center}

\vspace{4em}

\begin{center}
\textbf{\LargeОтчет по лабораторной работе} \\
\end{center}
\begin{center}
\textbf{\Large«Умножение матриц методом Фокса»} \\
\end{center}

\vspace{4em}

\newbox{\lbox}
\savebox{\lbox}{\hbox{text}}
\newlength{\maxl}
\setlength{\maxl}{\wd\lbox}
\hfill\parbox{7cm}{
\hspace*{5cm}\hspace*{-5cm}\textbf{Выполнила:} \\ студентка группы 381806-2 \\ Груздева Д.М.\\
\\
\hspace*{5cm}\hspace*{-5cm}\textbf{Проверил:}\\ доцент кафедры МОСТ, \\ кандидат технических наук \\ Сысоев А.В.\\
}
\vspace{\fill}

\begin{center} Нижний Новгород \\ 2021 \end{center}

\end{titlepage}

\setcounter{page}{2}

% Содержание
\tableofcontents
\newpage

% Введение
\section*{Введение}
\addcontentsline{toc}{section}{Введение}
\par Блочное умножение матриц
\newpage

% Постановка задачи
\section*{Постановка задачи}
\addcontentsline{toc}{section}{Постановка задачи}
\par В данном лабораторной работе требуется рассмотреть умножение матриц методом Фокса, реализовать последовательную
 и две параллельные версии этого метода (при помощи технологий OpenMP и TBB), провести эксперименты для оценки времени выполнения. 
 По полученным результатам сделать выводы.
\newpage

% Описание алгоритма
\section*{Описание алгоритма}
\addcontentsline{toc}{section}{Описание алгоритма}
Алгоритм Фокса основан на блочном разбиении матрицы между вычислителями.\par

Пусть  \( A= \left( a_{ij} \right) ,~~ B= \left( b_{ij} \right) ,~ i,j=\overline{0, n-1} \)  – квадратные матрицы размера  \( n \times n \) . Тогда:\par

 \[ C=A \times B,~ c_{ij}=a_{i0}b_{0j}+a_{i1}b_{1j}+ \ldots +a_{i,n-1}b_{n-1,j}= \sum _{k=0}^{n-1}a_{ik}b_{kj} \] \par

 \( c_{ij} \)  – результат скалярного произведения i-ой строки матрицы A и j-ого столбца матрицы B.\par

 Пусть число вычислителей равно 4, а n = 4, тогда блочное разбиение будет выглядеть следующим образом:\par

 \[ A_{00}= \left( \begin{matrix}
a_{00}  &  a_{01}\\
a_{10}  &  a_{11}\\
\end{matrix}
 \right) ,~~A_{01}= \left( \begin{matrix}
a_{02}  &   a_{03}\\
a_{12}  &   a_{13}\\
\end{matrix}
 \right)  \]  \[ A_{10}= \left( \begin{matrix}
a_{20}  &  a_{21}\\
a_{30}  &  a_{31}\\
\end{matrix}
 \right) ,~~A_{11}= \left( \begin{matrix}
a_{22}  &  a_{23}\\
a_{32}  &  a_{33}\\
\end{matrix}
 \right)  \] \par


\vspace{\baselineskip}
 Аналогично для B и C.\par

Сам алгоритм Фокса заключается в том, что на каждом шаге нужно выбрать подматрицу A в каждой строке, чтобы каждый вычислитель умножил полученную матрицу на имеющуюся подматрицу B, а затем отправил свою подматрицу B вышестоящему в схеме вычислителю.\par

Так как мы реализуем данный алгоритм на потоках, то нужды в передаче-получении подматриц у нас нет, что упрощает алгоритм.\par
\newpage

% Схема распараллеливания
\section*{Схема распараллеливания}
\addcontentsline{toc}{section}{Схема распараллеливания}

% Схема распараллеливания OpenMP
\subsection{OpenMP}
\addcontentsline{toc}{subsubsection}{OpenMP}
\newpage

% Схема распараллеливания TBB
\subsection{TBB}
\addcontentsline{toc}{subsubsection}{TBB}
\newpage

% Результаты экспериментов
\section*{Результаты экспериментов}
\addcontentsline{toc}{section}{Результаты экспериментов}
\begin{itemize}
\item Процессор: Intel Core i3-4130, 3400 MHz, физических ядер: 2, потоков: 4;
\item Оперативная память: 8192 МБ (DDR3), 800 MHz;
\item ОС: Microsoft Windows 10 Home, версия 2004 сборка 19041.572.
\end{itemize}

\par Для оценки времени работы рассмотренных выше реализаций сгенерируем случайные матрицы размерами 500x500, 1000x1000, 1500x1500
\par Результаты экспериментов представлены в Таблице 1.

\begin{table}[!h]
\caption{Результаты вычислительных экспериментов}
\renewcommand{\arraystretch}{1.8} 
\centering
\begin{tabular}{l*{4}{|l}}
\hline
& 500 & 1000 & 1500 \\
\hline
Размер матрицы & 0 & 0 & 0 \\
\hline
Наивная реализация & 0 & 0 & 0  \\
\hline
Последовательная блочная реализация & 0 & 0 & 0 \\
\hline
OpenMP & 0 & 0 & 0 \\
\hline
Ускорение OpenMP & 0 & 0 & 0 \\
\hline
TBB & 0 & 0 & 0 \\
\hline
Ускорение TBB & 0 & 0 & 0 \\
\end{tabular}
\end{table}

\par Из данных, представленных в Таблице 1 видно, что параллельная версия быстрее последовательной, а ее ускорение в общем случае пропорционально количеству запускаемых процессов. Такой хороший результат объясняется тем, что мы использовали относительно большое количество испытаний, из-за чего последовательная версия выполнялась гораздо дольше.
\newpage

% Заключение
\section*{Заключение}
\addcontentsline{toc}{section}{Заключение}
\par Главной задачей данной лабораторной работы была реализация параллельной версии, которая должна была быть эффективнее последовательной. Эта задача была успешно достигнута, о чем говорят результаты экспериментов, проведенных в ходе работы.
\par Кроме того, были разработаны и доведены до успешного выполнения тесты, созданные для данного программного проекта с использованием Google C++ Testing Framework и необходимые для подтверждения корректности работы программы.
\newpage

\end{document}